\documentclass{article}

\usepackage[top=1in,bottom=1in,left=1in,right=1in]{geometry}

\title{Interfacing With The HSSH}
\author{Collin Johnson}

\begin{document}

\maketitle

\section{Introduction}
This document discusses two things: the representation of the HSSH and our implementation of the HSSH. Each section
describes a layer of the HSSH, providing some motivation and theory behind the representation, a summarized description
of the representation, and details on how this representation is implemented in the Vulcan codebase.

The code for the HSSH is in the src/hssh/ directory of the Vulcan repository. Each layer of the HSSH has a subdirectory
within the src/hssh/ directory. The header files for the implementation of the representation of each layer exist
within this subdirectory. For example, the LPM is defined in src/hssh/local\_metric/lpm.h. Further implementation, like
the actual algorithm for LPM creation or the particle filters, are contained in further subdirectories to keep the
interface more apparent.

\section{Where Am I?}
The Hybrid Spatial Semantic Hierarchy (HSSH) is a hybrid metric/topological solution to the simultaneous localization
and mapping (SLAM) problem. The HSSH divides the SLAM problem into small-scale space and large-scale space. Small-scale
space is the portion of the world within the robot's immediate sensory horizon. Large-scale space deals with the world
as a whole, using information obtained from the small-scale space.

The HSSH is concerned with determining a map of the robot's environment and the position of the robot within this map. 
In the HSSH, a hierarchy of three spatial representations is used to describe the robot's environment and position. The
map built by the HSSH contains both metric and topological descriptions of the environment. Similarly, the position of
the robot is defined metrically and topologically. The spatial representations in the HSSH, referred to as layers of the
hierarchy, are the local metric, the local topological, and the global topological. Each layer provides a unique
description of the environment and the robot's position within the environment. The following sections detail how the
robot's map and position are described in each level of the HSSH.



% \begin{description}
%     \item[Local Metric:] \hfill
%         \begin{description}
%             \item[Pose:] relative to map reference frame
%         \end{description}
% 
%     \item[Local Topological:] \hfill
%         \begin{description}
%             \item[Local place:] small-scale place with topology described by small-scale star
% %             \item[Local path:] connection between local places
%         \end{description}
% 
%     \item[Global Topological:] \hfill
%         \begin{description}
%             \item[Global place:] large-scale place with topology described by large-scale star and incoming paths
%             \item[Global path segment:] connection between two global places along a path
%             \item[Global path:] larger structure connecting a series of places
%         \end{description}
% \end{description}
% 
% In the following sections, each quantity has an associated symbol, which will appear in parentheses next to the type.
% For example, \textbf{Pose(l):}.

\section{Local Metric}
The local metric layer is a metric description of the small-scale space around the robot. The map, referred to as the
local perceptual map (LPM), is a high-resolution occupancy grid. The position of the robot within the local metric
layer is a pose $(x,y,\theta)$ with respect to the reference frame of the LPM.

\subsection{Local Perceptual Map}
The local perceptual map (LPM) is a high-resolution metric map of the robot's local surround. The LPM represents the
part of the world within the robot's immediate sensory horizon and explicitly avoids handling large loop closures. The
LPM dynamically resizes as the robot explores its world, based on incoming sensor values. To maintain the invariant that
the LPM contains only the robot's immediate sensory horizon, the maximum dimension of the LPM is defined as twice the
maximum distance of a sensor reading. For a laser-based LPM, this size is twice the maximum usable scan range. In
addition to resizing based on sensory readings, the LPM scrolls as the robot moves, throwing away parts of the map that
fall outside the sensory horizon.

Maintaining a bounded-size LPM allows us to enforce two invariants. First, the robot is always well-localized within
the LPM. Limiting the size of the LPM to the the sensory horizon ensures that a good portion of the LPM is visible to
the robot at any time, which provides the observation model in the localization algorithm enough useful data to maintain
localization. Furthermore, restricting the size of the metric map used by the robot avoids large loop closures and
allows a simple SLAM algorithm to be used. In our implementation, the LPM is an occupancy grid. A particle filter is
used for localization, and the map is updated using the mean of the posterior probability distribution of the pose.

Second, computations using the LPM take a fixed amount of time. The LPM maintains an upper bound on its size,
guaranteeing a maximum running time of any operation that depends on the size of the map.

The LPM only guarantees local consistency, and the local frame of reference typically drifts over time, losing
metric accuracy with respect to a single global reference frame. For the HSSH, the lack of global consistency is not a
problem because global map consistency is maintained topologically in the global topological layer.

\subsection{Local Metric Representation}

\begin{description}
    \item[Pose($l,r$):] position, $(x,y,\theta)$, of the robot in the reference frame, $r$, of an LPM.

    \item[LPM($m$):] detailed map of the robot's local surround. An occupancy grid defined by the location of
    the bottom left cell (0,0) in the metric reference frame, the width and height measured in cells, and a conversion
    scale of meters/cell.
\end{description}

\subsection{Local Metric Implementation}
The code for the local metric layer exists in the src/hssh/local\_metric folder of the Vulcan repository. The
implementation of the LPM is a class LocalPerceptualMap (typedef'd to LPM) in the file lpm.h. The LPM implementation
provides easy access to the cost and classification for each cell in the occupancy grid.

The pose representation is a struct pose\_t defined in src/state/pose.h. pose\_t contains the $(x,y,\theta)$ of the
robot in the reference frame of the related LPM. The reference frame associated with the pose is implicit based on the
context in which the pose is used.

\section{Local Topological}
The local topological layer describes the qualitative decision structure of a local place. To determine the qualitative
decision structure, the local topological layer is required to perform place detection and then calculate the decision
structure of this place. Place detection and calculation of the decision structure depend on gateways, which define
transition points in the environment.

\subsection{Place Detection}

\subsection{Local Topological Representation}

\begin{description}
    \item[Metric place map($m_p$):] The metric place map is a high-resolution occupancy grid of the place. The boundary
    of the map is determined by ray tracing from the center of the gateways to occupied cells in the occupancy grid.
    Thus, the boundary is a combination of occupied cells and gateways.

    \item[Gateway($g_n$):] A gateway is a boundary between the place and either a path or another place. Each gateway
    defines the transition between the place and the rest of the environment. A gateway is represented by a line segment
    and a direction. The line segment is the location of the gateway in $m_p$. The direction points to the interior of
    the place.

    \item[Small-scale star($s$):] A small-scale star is a description of the topology of a local place. The small-scale
    star consists of an ordered sequence of local path fragments. There are two fragments per incident path, thus there
    are $2N_{path}$ local path fragments in the small-scale star.

    \item[Local path fragment($\pi_n$):] A local path fragment represents a path segment incident to the local place.
    The path fragment is associated with a gateway, $g_n$. Local path fragments come in pairs, defining the two
    directions, $+$ and $-$, of a path in the small-scale star. The directions are assigned arbitrarily. Each local path
    fragment has a flag indicating if the fragment is navigable or not.

    \item[Local place($p$):] A local place consists of a small-scale star and a metric place map. $p = (s, m_p)$
\end{description}

\subsection{Local Topological Implementation}
The code for the local topological layer exists in the src/hssh/local\_topological folder. Gateways are defined in
gateway.h as gateway\_t. small\_scale\_star.h contains SmallScaleStar and local\_path\_fragment\_t. The metric place map
is an LPM. The complete description of a place, a LocalPlace, is defined in local\_place.h.


\section{Global Topological}
The global topological layer is a topological description of the robot's large-scale environment. The representation
divides the world into places and paths. Places represent decision points in the world, locations where paths
intersect. Paths represent connections between places. Paths consist of an ordered sequence of path segments, where
each path segment represents the connection between two adjacent places. Thus, a path connects two or more places. The
topological map, then, is simply a set of places and a set of paths.

\subsection{Global Topological Representation}

\begin{description}
    \item[Large-scale star($S$):] The large-scale star represents the topology of a place. The star is an ordered
    sequence of global path fragments, analogous to the small-scale star. The fragments in the large-scale star
    correspond to global paths, though. Therefore, the large-scale star describes how global paths intersect at a place.
    
    \item[Global path fragment($\Pi_n$):] A global path fragment represents a path incident to a global place. The
    fragment contains the path id, the direction one would travel along the path if moving outbound through the
    fragment, and navigability flag indicating if travel is possible in the specified direction.
    
    \item[Global place($P$):] A global place is represented as a large-scale star and a unique identifier: $P = (id, S)$
    
    \item[Global path segment($\phi_{xy}$):] A global path segment represents the connection between two places. If the
    path segment is unexplored, one of these places will be labeled as a frontier. The path segment also contains a
    value, $\lambda_{xy}$, describing the transformation of the reference frame between the connected place,
    $[(0,0,0)_y]_x$. The path segment has a direction, so the places exist at the $+$ or $-$ end. Hence, the full path
    segment description is $\phi_{xy} = (P_x^+, P_y^-, \lambda_{xy})$.
    
    \item[Global path($\Phi$):] A global path represents an ordered sequence of connected places. The connection
    between consecutive places is represented as a global path segment. Thus, a global path is sequence of global path
    segments ordered head-to-tail. For example, $P_y^-$ of the first segment is $P_x^+$ of the second segment.
    
    \item[Topological map($T$):] The topological map is the collection of global places and global paths.
\end{description}

\subsection{Global Topological Implementation}
The global topological code exists in the src/hssh/global\_topological/ directory. large\_scale\_star.h contains
declarations for LargeScaleStar and global\_path\_fragment\_t. global\_place.h contains the declaration of GlobalPlace.
global\_path.h contains the declaration of GlobalPath. global\_path\_segment.h contains the declaration of
GlobalPathSegment. The topological map hypotheses are defined by the TopologicalMapHypothesis type declared in
topological\_map\_hypothesis.h, and the tree of topological maps is declared in tree\_of\_maps.h as TreeOfMaps.

\section{The You-Are-Here Pointer}
The complete description of the robot's position in the world consists of the robot's position at each of the layers of
the HSSH.

\subsection{You-Are-Here Representation}
\begin{description}
    \item[Local metric:] Location $l_m$ within the current LPM $m$.  $(l_m, m)$
    
    \item[Local topological:] Local place $p$ and the local path fragment through which the place was entered
    $\pi_{enter}$. $(p, \pi_{enter})$
    
    \item[Global topological:] The global place $P$ and the global path fragment through which the place was entered
    $\Pi_{enter}$, or the global path $\Phi$, current path segment $\phi_{xy}$, and direction of motion $+/-$. $(P,
    \Pi_{enter})$ or $(\Phi, \phi_{xy}, +/-)$
\end{description}

\subsection{You-Are-Here Implementation}
TODO

\section{Routes}
TODO

\end{document}
