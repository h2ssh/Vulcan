\documentclass{article}

\usepackage[top=1in,bottom=1in,left=1in,right=1in]{geometry}

\title{Vulcan Software Guide}
\author{Collin Johnson}

\begin{document}

\maketitle
\tableofcontents

\section{Introduction}

\section{Build System}

The Vulcan code uses SCons (http://scons.org/) for its build system.  The build system works just like any other 
hierarchical build system. In the root directory of the Vulcan project is a SConstruct file that sets up the build 
environment for the subdirectories in the project. Each subdirectory in the src/ folder has a SConscript file that 
handles the building for the utilities and modules that comprise the overall Vulcan project.

The SConstruct provides four environments for building:

\begin{itemize}
 \item \textbf{lib\_env}     : Used for building static libraries that will be used for building processes and module.
 \item \textbf{process\_env} : Used for building utility programs.
 \item \textbf{mod\_env}     : Used for building modules.
 \item \textbf{test\_env}    : Used for building unit tests with the Google Test library.
\end{itemize}

\textbf{process\_env} automatically links in the utils, math, and robot libraries which contain commonly used objects 
and functions, like angle calculations, shapes, poses, points, lines, and config files.

\textbf{test\_env} contains everything in \textbf{process\_env} and additionally links in the necessary libraries to 
support unit testing with Google Test. It includes -lgtest\_main, so a main.cpp isn't needed with a unit test program. 
This is described further in the Google Test documentation.

\textbf{mod\_env} extends the \textbf{process\_env} environment with the system and communication libraries.

Each module should create static libraries for help in building the code that interacts with other modules. The 
potential libraries to create are:

\begin{enumerate}
 \item A data library named \emph{module\_data} that contains all data types needed by other modules to function. For 
  example, the local\_metric\_hssh module outputs a LocalPose and LocalPerceptualMap to the other modules, so those are 
  the types included in its data library. Modules that use the LPM, like metric\_planner only need to link against 
  local\_metric\_hssh\_data to compile.
  
 \item A debug data library named \emph{module\_debug} that contains additional debugging outputs that aren't needed for 
  system functionality, just for debugging in the UI or testing programs.
  
 \item A functionality library named \emph{module\_func} that contains the additional code necessary for actually 
  running the processing that occurs inside the module. For example, if you want to calculate the isovists in the 
  \textsc{local\_topo\_hssh} in the DebugUI, you can link against   \emph{local\_topo\_hssh\_func} and that'll work. 
  This library is optional. If you want to test some things via the   DebugUI, then push those pieces into a func 
  library.
\end{enumerate}

Given these three libraries, building a module only requires compiling any remaining glue code, like data queues and the 
director, and linking them against these libraries. The DebugUI will needed to link with the data and debug libraries. 
Other modules should only need to link against the data library.

The reason for building in this fashion is that there are necessary interconnections between the modules. Finding a good 
level of granularity between the modules during the build process helps minimize the build time when changes are made.

\subsection{Static Library Dependencies}
Static libraries are just clumps of aggregated object files produced by the compiler. There is no linker step in their 
creation, any undefined symbols required in the object file must exist in another library or object file when a program 
is linked. In GCC, the order of static libraries matters for finding symbols. If library 'Foo' requires definition of a 
symbol 'bar', then 'bar' must be defined in a library that comes \emph{after} 'Foo' in the command-line arguments passed 
to GCC, i.e. if library 'Baz' defines 'bar', then the command-line for building test.cpp would be:

\begin{verbatim}
  g++ -o test test.cpp -lFoo -lBaz
\end{verbatim}

switching the order of Foo and Bar would result in an undefined reference linker error.

If there are circular dependencies between two static libraries, the code won't compile. They either need to be combined 
into a single library or refactored in some other way.

Great, so how does this apply to building modules and code with Vulcan and SCons? It's pretty simple. First, the 
\emph{module\_data}, \emph{module\_debug}, and \emph{module\_func} libraries must be a simple linear hierarchy of 
dependencies. Func can depend on Debug and Data. Debug can only depend on Data, but not Func. Data can't depend on 
either. Using this rule, when creating the environment for building, prepend the libraries needed by the module. The 
lowest-level Vulcan libraries are included by default. These libraries -- utils, math, robot, csp -- don't have 
dependencies on any other code, whereas most code uses at least something from one of these classes. By prepending, all 
the undefined symbols from high-level libraries will be correctly defined in the lowest-level libraries.

To take an example, the build process for \textsc{local\_topo\_hssh} uses the following static libraries and 
environments for building the module:

\begin{verbatim}

data_lib  = lib_env.Library(target = 'local_topo_hssh_data',  source = [data_objects])
debug_lib = lib_env.Library(target = 'local_topo_hssh_debug', source = [debug_objects])
func_lib  = lib_env.Library(target = 'local_topo_hssh_func',  source = [func_objects])

event_objects = mod_env.Object([Glob('events/*.cpp'),
                                'event_detector.cpp'])

app_objects = mod_env.Object(['director.cpp',
                              'main.cpp'])

localtopo_env = mod_env.Clone()

localtopo_env.Prepend(LIBS=['local_topo_hssh_func', 
                            'global_topo_hssh_data', 
                            'local_topo_hssh_debug', 
                            'local_topo_hssh_data', 
                            'local_metric_hssh_data', 
                            'hssh_utils'])
                            
local_topo_hssh = localtopo_env.Program('local_topo_hssh', [event_objects,
                                                            app_objects])
 
\end{verbatim}

\textsc{local\_topo\_hssh} uses all three libraries because the map editing tools need the full area segmentation code. 
The libraries are built first. Following this, an environment is cloned from the \emph{mod\_env} to add the specific 
libraries for \textsc{local\_topo\_hssh}. The libraries are prepended to the environment from highest-level to 
lowest-level. Ensuring the dependencies form a DAG makes specifying the order quite simple.

Each of the three libraries can then be used by other modules that use pieces of the \textsc{local\_topo\_hssh}. The 
DebugUI needs the data and debug libraries. The MapEditor needs all three libraries. The DecisionPlanner needs the data 
library.

\section{Basic Conventions and Types}
\subsection{Directories and Filenames}
\subsection{Reference Frames}
\subsection{Common Types}

\section{Configuration Files}

\section{Creating A Module}

\section{Creating A Message}

Inter-module communication on Vulcan uses a combination of two libraries: LCM and cereal. LCM is a UDP multicast-based 
message passing system. It was first used on MIT DARPA Urban Challenge car. It has nice utilities for logging, log 
playback, and viewing bandwidth usage. However, the need to maintain message definitions separately from the data types 
in the Vulcan code is cumbersome, error-prone, and time-consuming enough to discourage creation of messages that would 
otherwise be useful to have. 

To counteract the downside of LCM, Vulcan uses a single LCM message that contains a single buffer of bytes. On top of 
this message, the ModuleCommunicator class uses the cereal library and Boost.IOstreams to serialize \emph{any} Vulcan 
data type that satisfies two requirements:

\begin{enumerate}
 \item The type defines a private serialize method or a non-member serialize function as described in the documentation 
  for cereal: http://uscilab.github.io/cereal/ 
  
  \item The type specializes the message\_traits\_t struct to define which 
  network the message is sent on (System or Debug) and the channel(s) used for the type.
\end{enumerate}

The requirements for the serialize method are best learned by reading the cereal documentation. The gist is that all 
members of a class or struct are serialized into a binary archive. Each type being serialized must have a serialize 
method or function defined. All PODs and STL classes already have defined functions. Only the Vulcan types need to be 
defined. Many classes it simply comes down to listing the members, like below example with the pose\_t struct:

\begin{verbatim}

namespace vulcan
{
  namespace robot
  {
    
    template <class Archive>
    void serialize(Archive& ar, pose_t& pose)
    {
        ar (pose.x,
            pose.y,
            pose.theta);
    }
  }
}
    
\end{verbatim}

message\_traits\_t defines the properties of a type needed for transmission and subscription via the ModuleCommunicator. 
There are two macros defined that make the specialization very simple. They take the type and then a paren-enclosed list 
of channel names. The vast majority (everything but polar\_laser\_scan\_t?) of types have a single channel, but the 
parens are still required for the macro to work correctly.

\begin{verbatim}
   
    DEFINE_SYSTEM_MESSAGE(robot::pose_t, ("HSSH_LOCAL_POSE"))
    DEFINE_DEBUG_MESSAGE(robot::pose_t, ("HSSH_LOCAL_POSE"))
    
\end{verbatim}

A few things to keep in mind about these macros:

\begin{enumerate}
 \item The macro must appear in the header that declares the type being serialized. 
 \item The macro must be declared in the global namespace.
 \item The vulcan namespace isn't needed when specifying the type.
 \item A message macro should only be written for types intended to be sent between modules. Internal types, like 
  math::Point, don't have an associated macro because they aren't themselves sent between modules. They can have 
  serialize functions though, so they can be easily serialized inside other types that will be sent between modules.
\end{enumerate}

There are two networks on which messages are sent, Debug and Release. The messages are split between the networks 
because it is more efficient for processing. In particular, with UDP multicast, any message sent is received by 
\emph{all} modules listening on that address. If you send 10Mb/s in maps, those 10Mb/s are received by every process 
listening, though not transmitting, on that network. While the most efficient network would define a separate address 
for each message, that, like the build process, is too fine a granularity to sanely maintain, so we define two networks.

The \textbf{System} network is for messages essential to the functioning of Vulcan. These are the defined outputs of a 
module that are required inputs for other modules to function. For example, the \textsc{local\_metric\_hssh} modules 
outputs LocalPerceptualMap, LocalPose, and DynamicLaserPoints. These are used by myriad other modules. 

The \textbf{Debug} network is for messages intended for the DebugUI for visualizations, for the TestRunner for 
integration testing, or some other logging facility. They aren't required for the functionality of the robot and thus 
are only read by the DebugUI under normal circumstances. Again using \textsc{local\_metric\_hssh}, it outputs particle 
filter information, lines extracted from laser scans, and other internal details that highlight the processing of the 
module, but aren't needed by other modules.

\subsection{Polymorphic Types and Static Libraries}
The serialization code will work with polymorphic types, as described at \\ 
http://uscilab.github.io/cereal/polymorphism.html. However, it requires a little extra undocumented work in order to 
actually get it functioning. In particular, the linker seems to enjoy throwing away the generated serialization 
functions. Thus, the code will compile, but the desired functions won't be defined on the other side and a run-time 
exception will be thrown. It will probably look like:

\begin{verbatim}
 EXCEPTION: Failed to unserialize message of type: 
 N6vulcan4hssh46global_metric_relocalization_request_message_tE
 Error : Trying to load an unregistered polymorphic type 
 (vulcan::hssh::FreeSpaceFilterInitializer) Message discarded!
\end{verbatim}

In order to make the linker play nice, you need to do the following at the bottom of any header that defines a 
polymorphic type:

\begin{verbatim}
// Serialization support for smart pointers
#include <cereal/archives/binary.hpp>
#include <cereal/types/polymorphic.hpp>

CEREAL_REGISTER_TYPE(fully-qualified type)
\end{verbatim}

Additionally, the header for the concrete type must be included somewhere in the module being compiled. If it isn't 
included, then the registration never happens, make sense?

For an example of this code in action, see hssh/local\_metric/director.cpp and \\
hssh/local\_metric/commands/rotate\_lpm.h.

\section{Adding Debugging Visualizations}


\section{Testing}

\subsection{Static Analysis}

Static analysis is useful for helping find errors or inconsistencies lurking in the code. Clang has a good static 
analyzer included with its distribution.

\section{Bug Reporting}
    
    
\end{document}
